% 第8章 おわりに
\newpage
\renewcommand{\baselinestretch}{1.5}
\section{おわりに}
\renewcommand{\baselinestretch}{1}
\par 本研究では、ウェブページのレイアウトに着目して要素単位での顕著度を計測することでウェブページに特化した重要領域の視覚化手法を提案した。まず初めにパブリック上に存在しなかったモダンデザインを含めたウェブページのオリジナル視線データセットを作成した。このデータセットでは合計270個のウェブページを使用して35名の被検者の
視線データを収集した。また、被験者の視線データを分析することでウェブページ固有の視線のバイアスやレイアウトによる傾向などを新たに確認することができた。

\par オリジナルデータセットを用いて取得したウェブページの視線データセットを用いてウェブページ固有の要素に着目して要素単位のレイアウトを考慮した正確な顕著度の推定を行った。さらに要素ごとの顕著度を可視化する為に提案手法である顕著領域マップの生成を行った。提案手法の顕著領域マップの顕著度推定精度は他のニューラルネットワークモデルと比較して高く、重要領域の見やすさについても既存の顕著性マップと比較して認識しやすいことが明らかになりウェブページに最適な顕著度可視化マップであると言える。これによりデザイナーはユーザーがウェブページを閲覧した際にどの要素に注目する可能性が高いのかをウェブページの開発段階で正確に判断する事が可能となる。また、それに合わせて注目してほしい情報を上手に配置することで、その情報にユーザーが注目しやすくなり効率的なユーザーの獲得につながるのではないかと考えられる。

\par さらに、計算された要素単位での顕著度を用いて顕著度のランキングを作成した。このランキングを使用することで特に顕著度が高い要素をタイル状に並べて一つの画像で表した集約図の生成を行なった。集約図についてはウェブページの内容理解面で詳細内容までは理解することが難しいなどの課題が残る結果となった。これらの顕著度推定後の可視化面でより見やすいレイアウトなどの工夫を行うことでウェブページの内容理解支援ツールとして改善できると考える。\\

\par 本研究の今後の課題は以下の通りである。

\begin{itemize}
  \item 顕著領域マップの見せ方の改善
  \par ~~本手法では要素単位で顕著度に応じた明度で塗り潰すことで顕著領域マップの生成を行った。しかしながら、被験者の意見の中に元のウェブページと照らし合わせる必要があるなどの視認性の問題が挙げられた。これらの問題を解決する為にGUI上で元のウェブページとボタンで切り替えるなどの見せ方の改善を行うことでより使いやすく分析しやすいものに改善できると考えている\\

  \item 集約図の見せ方の改善
  \par ~~本手法で被験者から最も否定的な意見が多かったのが顕著度の高い重要領域をまとめた集約図の見せ方についてである。ウェブページの内容理解支援ツールを目的として集約図を提案したが、ページ内容の大枠程度しか理解できずあまり効果的でないという意見が目立った。これらを解決する為、集約図のレイアウトや表示する要素数の変更などを行いウェブページの内容を理解しやすい物へと改善したいと考える。また、過去に研究発表した「ワードクラウドのグラデーション描写による多次元化」\cite{inagaki2018}の技術と組み合わせる事で画像だけでなくウェブページのテキストも合わせて分析した手法も可能か検討したいと考えている。
  \\
\end{itemize}