% 第7章 おわりに
\newpage
\renewcommand{\baselinestretch}{1.5}
\section{おわりに}
\renewcommand{\baselinestretch}{1}
\par 本研究では、要素単位で顕著度を判断する顕著領域マップの生成を行った。これによりデザイナーはウェブページの開発段階で、ユーザーがウェブページを閲覧した際にどの要素に注目する可能性が高いのか正確に判断する事が可能となる。また、それに合わせて注目してほしい情報を上手に配置することで、その情報にユーザーが注目しやすくなり効率的なユーザーの獲得につながるのではないかと考えられる。さらに、計算した顕著度を元に重要度が高い領域を一つの画像にまとめた集約図の生成を行なった。これにより初見のウェブページの内容理解を支援する事が可能となる。


\par 本手法は、自然画像には存在しないウェブページ特有の要素単位で顕著度を計算した後に顕著度に応じた色で要素を塗りつぶす事で顕著領域マップの生成を行った。評価では、本手法が顕著度を計算して出力した顕著度ランキングの精度はある程度良く、重要領域の見やすさについても既存の顕著性マップと比較して見やすい事が分かった。また、重要領域を一つの画像にまとめた集約図においては、集約図を見る事でページ内容をある程度は判断できるものの非常に効果があるとは言えないことが評価実験から明らかになり、ウェブページの内容理解支援ツールとしては今後改善が必要であると言える。\\


\par 本研究の今後の課題は以下の通りである。

\begin{itemize}
    \item 顕著性マップ生成モデルの変更による顕著度計算精度の向上
    \par ~~本手法では、顕著性マップの生成に自然画像向けのItti-Kochらの顕著性マップ生成モデルを使用したが、ウェブページに適した精度の高い顕著性マップ生成モデルを使用する事で重要領域の抜き出し精度が更に向上すると考えられる。 \\

    \item 最上部だけでなくページ全体の顕著度が高い要素の集約図の生成
    \par ~~本手法において、ウェブページの最上部の領域のみを解析する事で重要度が高い要素を集約した集約図を生成した。しかしながら、最上部だけでなくウェブページの最下部までの全領域の要素を解析してページ全体の集約図を生成する事でよりウェブページの内容が分かりやすい物となると考えている。 \\

    \item テキスト内容要約手法と組み合わせたウェブページの内容要約の視覚化手法の研究
    \par ~~過去に研究発表した「ワードクラウドのグラデーション描写による多次元化」\cite{inagaki2018}と組み合わせる事で、ウェブページの構造だけでなく、文章内容も考慮したウェブページの要約視覚化手法を考案出来るのでは無いかと考えている。 \\
\end{itemize}